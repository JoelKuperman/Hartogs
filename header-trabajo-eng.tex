\usepackage[unicode,hyperfootnotes=false,hidelinks,hyperindex=false]{hyperref}
\usepackage{graphicx,tikz,framed,amssymb,multicol,enumerate,xcolor}
\usepackage{amsmath,amsthm,amsfonts,amssymb,xcolor,etoolbox,cancel}
\usepackage[english]{babel}
%%\usepackage{postnotes}%%
\usepackage[numbers]{natbib}
\usepackage[T1]{fontenc}
\usepackage{dsfont}
\graphicspath{ {./logos/} }
\usepackage[export]{adjustbox}


%%%%%%%%%%%%% Postnotes %%%%%%%%%%%%%%%%
%%\postnotesetup{heading=\chapter*{Notas},format=}%%

%%%%%%%%%%%%%% Endnotes %%%%%%%%%%%%%%%%%%
%% \usepackage{endnotes,endnotes-hy}
%% \renewcommand{\enotesize}{\normalsize}
%% \renewcommand{\notesname}{Notas}

%%%%%%%%%%%%% Footnotes %%%%%%%%%%%%%%%%
\usepackage[symbol]{footmisc}
%% \renewcommand*{\thefootnote}{\fnsymbol{footnote}} % deja solo asteriscos!
\usepackage{perpage}
\MakePerPage{footnote} % resetea enumeración por páginas

%%%%%%%%% Equation numbering %%%%%%%%%%%%
\numberwithin{equation}{section}


%flechas extensibles
\usepackage[arrow,matrix]{xy}
\newlength{\xywd}
\newcommand{\xymapsto}[2][]{\mathrel{%
  \sbox{0}{$\scriptstyle#1$}%
  \xywd=\wd0
  \sbox{0}{$\scriptstyle#2$}%
  \ifdim\wd0>\xywd \xywd=\wd0 \fi
  \xymatrix@M=0ex@C\dimexpr\xywd+0.7em\relax{{}\ar@{|->}[r]^{#2}_{#1}&{}}%
}}
\newcommand{\xyrightarrow}[2][]{\mathrel{%
  \sbox{0}{$\scriptstyle#1$}%
  \xywd=\wd0
  \sbox{0}{$\scriptstyle#2$}%
  \ifdim\wd0>\xywd \xywd=\wd0 \fi
  \xymatrix@M=0ex@C\dimexpr\xywd+0.7em\relax{{}\ar@{->}[r]^{#2}_{#1}&{}}%
}}

\newcommand{\N}{\mathbb{N}}
\newcommand{\C}{\mathbb{C}}
\newcommand{\R}{\mathbb{R}}
\newcommand{\Z}{\mathbb{Z}}
\newcommand{\Q}{\mathbb{Q}}
\newcommand{\op}{\operatorname}
%% Parte entera
\newcommand{\entera}[1]{\lfloor#1\rfloor}
\newcommand{\defi}{\mathrel{\mathop:}=}
\newcommand{\fmaps}{\xymapsto{f}}
\newcommand{\gmaps}{\xymapsto{g}}
\newcommand{\True}{\textbf{V}}
\newcommand{\False}{\textbf{F}}
\newcommand{\Undef}{\textbf{X}}
\newcommand{\ent}{\Rightarrow}
\newcommand{\tne}{\Leftarrow}
\newcommand{\sm}{\smallsetminus}
\newcommand{\citecut}[1][\textup{\textellipsis\!}]{\textup{[}#1{\textup{]}}}
\newcommand{\et}{\mathrel{\,\&\,}}
\newcommand{\union}{\mathop{\textstyle\bigcup}}
\newcommand{\bigunion}{\mathop{\raisebox{0ex}[0ex][0.6ex]{$\displaystyle\bigcup$}\vspace{0ex}}}
\newcommand{\sbq}{\subseteq}
\newcommand{\nsbq}{\subseteq}
\newcommand{\Pow}{\mathop{\mathcal{P}}}
\newcommand{\lb}{\langle}
\newcommand{\rb}{\rangle}
\newcommand{\Elo}{\rotatebox[origin=c]{180}{E}loísa}
\newcommand{\Abe}{\rotatebox[origin=c]{180}{A}belardo}
\newcommand{\Caso}[1]{\noindent\fbox{Caso #1.}\ }
\newcommand{\suces}[2][n]{\left\{ {#2}_{#1} \right\}_{#1\in\N}}
\newcommand{\sucesf}[2][n]{\left\{ #2 \right\}_{#1\in\N}}


%% Nuevos "tiende a"
\newcommand{\subea}{\mathrel{\text{\raisebox{-0.4ex}[0ex][0ex]{\rotatebox{20}{\hspace{-0.2ex}${\to}$\hspace{-0.25ex}}}}}}
%\newcommand{\subea}{\mathrel{\text{\raisebox{0.5ex}[0ex][0ex]{\rotatebox{-25}{$\nearrow$}}}}}
\newcommand{\bajaa}{\mathrel{\text{\raisebox{0.5ex}[0ex][0ex]{\rotatebox{-25}{\hspace{-0.1ex}${\to}$\hspace{-0.1ex}}}}}}

%% Definición temporaria
\newcommand{\azul}[1]{\textbf{\mbox{$#1$}}}

\DeclareMathOperator{\sen}{sen}
\DeclareMathOperator{\dom}{dom}
\let\Im\undefined
\DeclareMathOperator{\Im}{Im}
\DeclareMathOperator{\ran}{ran}
\DeclareMathOperator{\alt}{alt}
\DeclareMathOperator{\type}{type}
\DeclareMathOperator{\cat}{cat}
\DeclareMathOperator{\bin}{bin}
\DeclareMathOperator{\cf}{cf}

%Ya sé que quedan feos pero me daba paja tipear eso de nuevo
\DeclareMathOperator{\Bst}{\text{BST}^{--}}
\DeclareMathOperator{\Bsti}{\text{BST}^{--}\hspace{-2 pt}\text{+I}}
\DeclareMathOperator{\WO}{WO}

\newtheorem{theorem}{Theorem}[section]
\newtheorem{lemma}[theorem]{Lema}
\newtheorem{prop}[theorem]{Proposition}
\newtheorem{corollary}[theorem]{Corollary}
\newtheorem{problem}[theorem]{Problema}
\theoremstyle{definition}
\newtheorem{definition}[theorem]{Definition}
\newtheorem{example}[theorem]{Ejemplo}
\newtheorem{exercise}[theorem]{Ejercicio}
\newtheorem*{exercise*}{Ejercicio}
\theoremstyle{remark}
\newtheorem{claim}{Claim}
\newtheorem*{claim*}{Claim}
\newtheorem*{remark*}{Remark}
\newtheorem*{ayuditaaa*}{Ayuda}

%Defino un entorno para notaciones, ver que no dé problemas

\newtheorem*{notation}{Notación}
\theoremstyle{remark*}
\newtheorem{remark}{Observación}

\newenvironment{hint*}{\pushQED{\qed}\begin{ayuditaaa*}}{\popQED\end{ayuditaaa*}}
\newtheorem*{solveitnow*}{Solución} % \newtheorem establishes the object heading
\newenvironment{solution}    % this is the environment name for the input
{%
  \pushQED{\qed}\begin{solveitnow*}}
{\popQED\end{solveitnow*}}

%
%%%%%%%%%%%%%%%%%%%%%  ADOBE FONTS  %%%%%%%%%%%%%%%%%%%%%%%%%%%%%%%%%%%%%%%


%%%%%%%%%%%%%%%%%%%  END ADOBE FONTS  %%%%%%%%%%%%%%%%%%%%%%%%%%%%%%%%%%%%%
%

%%%%%%%%%%%%%%%%%%%%%%%%%%  TOCLOFT  %%%%%%%%%%%%%%%%%%%%%%%%%%%%%%%%%%%%%%
%% scrartcl incompatibility ahead but best solution so far
\usepackage{tocloft}
\newlength{\imarg}

\newcommand{\marg}{\hspace*{0.2em}}
\setlength{\imarg}{0.2em}

\renewcommand{\cfttoctitlefont}{\hfill\normalsize\bfseries}
\renewcommand{\cftaftertoctitle}{\hfill\hspace{0em}}
\renewcommand{\cftsecafterpnum}{\marg}
\renewcommand{\cftsubsecafterpnum}{\marg}
\renewcommand{\cftsubsubsecafterpnum}{\marg}
\setlength{\cftsecindent}{\imarg}
\setlength{\cftsubsecindent}{\imarg}
\setlength{\cftsubsubsecindent}{\imarg}
%%%%%%%%%%%%%%%%%%%%%%%  END TOCLOFT  %%%%%%%%%%%%%%%%%%%%%%%%%%%%%%%%%%%%%
%%
%%
\newcommand{\Var}{\mathit{Var}}
\newcommand{\fixref}[1]{\ref*{#1}\relax{}}
\newcommand{\ayuda}[1]{\hyperlink{#1}{\modley}}
\newcommand{\solucion}[1]{\hyperlink{#1}{$\circledS$}}
%
%%%%%%%%%%%%%%%%%%%%%%%  INDEXING     %%%%%%%%%%%%%%%%%%%%%%%%%%%%%%%%%%%%%

\usepackage[nottoc]{tocbibind}  %% índice en TOC
\usepackage{makeidx}
%\usepackage{index}

%%%%%%%
%% Versión con links que apuntan al lugar (código de
%% http://tex.stackexchange.com/a/77656/69595)
\newcounter{indexanchor}

%% This code is mine
\makeatletter
\long\def\raiseboxzero#1#2{%
  \@begin@tempboxa\hbox{#2}%
    \setlength\@tempdima{#1}%
    \setlength\@tempdimb{0pt}%
    \setbox\@tempboxa\hbox{\raise\@tempdima\box\@tempboxa}%
    \ht\@tempboxa\@tempdimb%
    \box\@tempboxa
  \@end@tempboxa}
\makeatother

\newcommand*{\xindex}[1]{%
  \stepcounter{indexanchor}% make anchor unique
  \def\theindexterm{#1}%
  \edef\doindexentry{\noexpand\index
    {\expandonce\theindexterm|indexanchor{index-\theindexanchor}}}%
  \raiseboxzero{\baselineskip}{\hypertarget{index-\theindexanchor}%
    {\doindexentry}}%
}

\newcommand*{\indexanchor}[2]{\hyperlink{#1}{#2}}

%% Me gustaba más \textsl abajo pero no está en newtx
\makeatletter
\newcommand{\ntrm}{\@dblarg\@mycommand}
\def\@mycommand[#1]#2{\textit{\textbf{#2}}\xindex{#1}}
\makeatother
%%
\makeindex
%%%%%%%%%%%%%%%%%%%%%%%  END INDEXING  %%%%%%%%%%%%%%%%%%%%%%%%%%%%%%%%%%%%

%% Corrección de tikz por Babel
\tikzset{
every picture/.append style={
  execute at begin picture={\deactivatequoting},
  execute at end picture={\activatequoting}
  }
}

%% HyperRef config (despues de headers)
%%
\hypersetup{
  pdftitle={Teoría de Conjuntos: Árboles conjuntistas y su relación con el problema de Suslin},
  pdfauthor={Gervasio Figueroa},
  pdfsubject={Set Theory},
  pdfkeywords={axiomas, funciones, límites, sucesiones, derivadas},
  colorlinks,
  linkcolor={blue!35!black},
  citecolor={blue!35!black},
  urlcolor={blue!35!black}
}

%%% Local Variables:
%%% mode: latex
%%% TeX-master: "trabajo_final_figueroa"
%%% ispell-local-dictionary: "castellano8"
%%% End:
