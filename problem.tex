\section{Problem}
\begin{definition}\label{aleph}
	Let $X$ be any set,
	\[
	\text{WO}(X):=\{\langle A,R\rangle\ |A\subset X \ \text{is well-ordered by}\ R\} := \text{W}, 
	\]
	and $\aleph(X):= \text{W}/{\sim_X}$ where $\langle A,R\rangle \sim_X \langle B, R'\rangle \Longleftrightarrow \langle A, R\rangle =_o \langle B, R'\rangle$ and $=_o$ denotes initial similarity (Moschovakis, def. 7.27 \cite{Moscho}).
	Given $\aleph(X)$, and $A,B \in\aleph(X)$ we shall say $A \leq_{\aleph(X)} B$ iff for any (all) $U\in A$ and $V\in B$ it holds that $U\leq_o V$, where this last relation means $U$ is isomorphic to an initial segment of $V$.
\end{definition}

\begin{lemma}\label{aleph2}
	For all $X$, $\aleph(X)$ has the following properties:
	\begin{enumerate}[(i)]
		\item $\aleph(X)\nleq_c X$
		\item $(\aleph(X),\leq_{\aleph(X)})$ is a well order and is not equinumerous with any of its initial segments.
	\end{enumerate}
\end{lemma}

%\begin{claim}
%   $\aleph(X)$ is a cardinal and $\aleph(X)\nleq_c X$
%\end{claim}

\begin{theorem}
	For any set $X$, $\aleph(X)<_c \mathcal{P}^4(X)$.
\end{theorem}

\begin{proof}
	We will show that there's an injective function from $\aleph(X)$ into $\mathcal{P}^4(X)$. Let $f:WO(X)\rightarrow \mathcal{P}^3(X)$ defined by $f(\langle A,R\rangle) = \{\{\{a\},\{a,b\}\}| a\leq_R b\}$. Note that $f$ is injective. Now, we can define an injective function $\Tilde{f}: \mathcal{P} (WO(X)) \rightarrow \mathcal{P}^4(X)$ given by $\Tilde{f}(S) = \{f(s) | s\in S\}$. This shows that $\mathcal{P} (WO(X)) \leq_c \mathcal{P}^4(X)$. Hence, since $\aleph (X) \subset \mathcal{P} (WO(X))$, we obtained what we wanted to show.
\end{proof}

\begin{theorem}\label{partes3}
	For any set $X$, $\aleph(X)<_c \mathcal{P}^3(X)$.
\end{theorem}

% Ver si es consistente con ZFC que valga $\aleph(X)<_c \mathcal{P}(X)$ (es consistente el igual en todos lados, ver si pueden fallar todas las instancias de GCH).
% GCH puede fallar en todos lados: https://www.jstor.org/stable/2944324
% Ver si se puede probar en ZFC que $\aleph(X)<_c \mathcal{P}^2(X)$. Prueba rapida: $|X| = \kappa$, $\aleph(X) = \kappa^{+} \leq_c 2^{\kappa} <_c 2^{2^{\kappa}} = |\mathcal(P)^{2}(\kappa)|$.
% ¿Es consistente en BST que no valga con $\mathcal(P)^{2}$ para algun X?

\begin{proof}
	%	We will show that there's an injective function from $\aleph(X)$ into $\mathcal{P}^4(X)$. Let $f:WO(X)\rightarrow \mathcal{P}^3(X)$ defined by $f(\langle A,R\rangle) = \{\{\{a\},\{a,b\}\}| a\leq_R b\}$. Note that $f$ is injective. Now, we can define an injective function $\Tilde{f}: \mathcal{P} (WO(X)) \rightarrow \mathcal{P}^4(X)$ given by $\Tilde{f}(S) = \{f(s) | s\in S\}$. This shows that $\mathcal{P} (WO(X)) \leq_c \mathcal{P}^4(X)$. Hence, since $\aleph (X) \subset \mathcal{P} (WO(X))$, we obtained what we wanted to show.
	%\end{proof}
	
	Let $f:WO(X)\rightarrow \mathcal{P}^2(X)$ defined by $f(\langle A,\mathrel{R}\rangle) = \{\{b\mid b \mathrel{R} a\} \mid a\in A\}$. $f$ is clearly injective and an argument similar as before yields what we wanted to show.
\end{proof}

\begin{theorem}[ZFC]
	There is no sequence $\{X_n\}_{n\in\omega}$ such that for all $n\in\omega$ $X_{n+1} <_c X_n$.
\end{theorem}
\begin{proof}
	Assume by contradiction that there is such a sequence, and let us take $\{\alpha_n\}_{n\in\omega}$ such that each $\alpha_n$ is an ordinal with the same cardinality as $X_n$. We then have an infinite sequence of ordinals of decreasing cardinality, which leads us to a contradiction.
\end{proof}

% Ver si en ZF, el teorema anterior es equivalente a AC. AC lo implica, falta ver la vuelta.
% Se llama NDS el teorema anterior.
% Lectura inicial:  https://link.springer.com/article/10.1007/s00153-015-0472-5#:~:text=Abstract,m%E2%80%9D%20%E2%86%9B%20NDS%20in%20ZFA

The previous result relies heavily on Choice, in fact, given any countable partially ordered set $\langle P,< \rangle$ it is consistent with ZF for there to be a family of sets $\mathcal{F}$ such that $\langle \mathcal{F},<_c\rangle$ is isomorphic to $\langle P,< \rangle$.
%CONSEGUIR CITA DE LO ANTERIOR
%También, ver si puede valer el teorema anterior y fallar Elección, posiblemente salga de
%estudiar los modelos en los que falla.

We shall use, following the convention from \cite[Def. I.3.1]{Kunen}, $\Bst$ to mean the axioms of Extensionality, Comprehension, Pairing, Union and Powerset. 
Let us use $\Bsti$ to denote $\Bst$ with the axiom of Infinity.

%Ver si se puede sacar Infinito sin problemas

%Kunen da la conjunción de Partes of Reemplazo.

\begin{theorem}[$\Bst$]
	There is no sequence $\{X_n\}_{n\in\omega}$ such that for all $n\in\omega$ $\mathcal{P}(X_{n+1}) <_c X_n$.
\end{theorem}

Before moving on to the proof, let us note that Definition \ref{aleph} does not rely on any axiom outside of $\Bst$, with the caveat that without Replacement, $\aleph(X)$ cannot be proven to be isomorphic to an ordinal.

\begin{proof}
	Assume there is such a sequence, $\mathcal{S} := \{X_n\}_{n\in\omega}$, then $\Bst$ proves that $\ran(\mathcal{S})$ exists (\cite[Def. I.6.6]{Kunen}).
	Moreover, for all $X\in\ran(\mathcal{S})$: 
	\[
	\aleph(X) \subset \mathcal{P}(\WO(X)) \subseteq \mathcal{P}(\WO(\union \ran(\mathcal{S}))),
	\]
	where both inclusions follow from Definition \ref{aleph}.
	
	\noindent Thus, for all $X\in\ran(\mathcal{S})$, we have $\aleph(X)\in\mathcal{P}(\mathcal{P}(\WO(\union \ran(\mathcal{S}))))$. Hence, $\Bst$ proves the existence of the sequence $\{\aleph(X_n)\}_{n\in\omega}$, which would of course be trivial using Replacement.
	
	It is straightforward to prove that for each $n\in\omega$, 
	$\mathcal{P}^{3}(X_{n+3}) <_c X_n$, 
	indeed, applying the hypothesis on $\mathcal{S}$ yields:
	\begin{align*}
		&\mathcal{P}(X_{n+3}) \hspace{4.5 pt} <_c X_{n+2}
		\\ \implies &\mathcal{P}^2(X_{n+3}) \leq_c \hspace{4.5 pt} \mathcal{P}(X_{n+2}) <_c X_{n+1}
		\\ \implies &\mathcal{P}^3(X_{n+3}) \leq_c \mathcal{P}^{2}(X_{n+2}) \leq_c \mathcal{P}(X_{n+1}) <_c X_n.
	\end{align*}
	As a consequence,
	$\aleph(X_{n+3}) <_c \mathcal{P}^{3}(X_{n+3}) <_c X_n$, where the first inequality comes from applying Theorem \ref{partes3}.
	
	\noindent Let us now show that $\aleph(X_{n+3}) <_c \aleph(X_n)$ for all $n\in\omega$. Well ordered sets are pairwise comparable with the relation $\leq_c$, in particular: $\aleph(X_{n+3}) <_c \aleph(X_n)$ or $\aleph(X_n) \leq_c \aleph(X_{n+3})$. The latter statement cannot hold, since it would imply $\aleph(X_n) <_c X_n$, which contradicts Lemma \ref{aleph2}.
	
	Finally, $\{\aleph(X_{3n})\}_{n\in\omega}$ is a set of well ordered sets that is not well founded with the $<_c$ relation, which cannot happen under $\Bst$.
\end{proof}

The following theorem is equivalent to the previous one when using Replacement but strictly stronger when working within $\Bst$.

\begin{theorem}[$\Bst$]
	\label{hartogs_class}
	Let $\Phi(\cdot,\cdot)$ be a formula with two free variables, then $\Phi$ cannot satisfy both of the following properties:
	\begin{enumerate}[(i)]
		\item \label{nat_define} For each natural number $n$ there is exactly one set $X$ that satisfies $\Phi(n,X)$.
		\item \label{dec_seq} For every $n\in\omega$ the following holds: 
		\[
			(\Phi(n,X) \land \Phi(n+1,Y)) \implies \mathcal{P}(Y) <_c X.
		\] 
	\end{enumerate}
\end{theorem}

Before giving a proof for Theorem \ref{hartogs_class} we note the following result from \cite[Corollary 7.33]{Moscho}.

\begin{prop}\label{class_wo}
	Every non-empty class $\mathcal{C}$ of well ordered sets has a $\leq_o$-least member, i.e., for some $U_0 \in \mathcal{C}$ and all $U\in\mathcal{C}$, $U_0$ is isomorphic to an initial segment of $U$.
\end{prop}

Given that without the Axiom of Infinity $\omega$ could be a proper class, when we say $n\in\omega$ we shall really mean $\forall x\in n \ (x = \emptyset \lor \exists y \in n \ x = S(y))$, where $S(a) = a \cup \{a\}$.

\begin{proof}[Proof of Theorem \ref{hartogs_class}]
	Let $\Phi$ be a formula with two free variables and assume it satisfies (\ref{nat_define}) and (\ref{dec_seq}), we shall take the following class:
	\[
		 \mathcal{C} := \{X: \exists Y ((\exists n\in\omega \ \Phi(n,Y)) \land X = \aleph(Y))\}.
	\]
	By (\ref{nat_define}), Definition \ref{aleph} and Lemma \ref{aleph2}, $\mathcal{C}$ is a non-empty class of well ordered sets. By Proposition \ref{class_wo}, there must be a $\leq_o$-least member of $\mathcal{C}$.
	Let $X_0$ be such a member of $\mathcal{C}$, then there must be some $Y_0$ and $n_0 \in\omega$ such that $X_0 = \aleph(Y_0) \land \Phi(n_0,Y_0)$.
	The assumption (\ref{nat_define}) implies there are $Y_1 , Y_2, Y_3$ such that $\Phi(n_0 + i , Y_i)$ for $i\in\{1,2,3\}$.
	Applying (\ref{dec_seq}), we get:
	
	\begin{align*}
		&\mathcal{P}(Y_3) \hspace{4.5 pt} <_c Y_2
		\\ \implies &\mathcal{P}^2(Y_3) \leq_c \hspace{4.5 pt} \mathcal{P}(Y_2) <_c Y_1
		\\ \implies &\mathcal{P}^3(Y_3) \leq_c \mathcal{P}^{2}(Y_2) \leq_c \mathcal{P}(Y_1) <_c Y_0.
	\end{align*}
	Recall from Theorem \ref{partes3} that $\aleph(Y_3) <_c \mathcal{P}^{3}(Y_3)$, and using the inequality above, we obtain $\aleph(Y_3) <_c Y_0$. But $\aleph(Y_3)\in\mathcal{C}$, which contradicts $X_0$ being $\leq_o$ minimal. 
\end{proof}

DUDA IMPORTANTE, ¿CUÁL ES EL ALEPH DE UN CONJUNTO DEDEKIND FINITO (NO FINITO)? ¿SIEMPRE VA A SER UN BUEN ORDEN FINITO? EN CASO CONTRARIO, EN UN MODELO DE BST SIN EL AXIOMA DE INFINITO TIENE QUE EXISTIR ESE ALEPH, ¿PERO EL HECHO DE QUE HAYA UN BUEN ORDEN INFINITO NO IMPLICA QUE $\omega$ EXISTE?

OTRA DUDA, ¿EN $\Bst$ VALE EL SIGUIENTE TEOREMA? ES SIQUIERA ENUNCIABLE?

\begin{theorem}[$\Bst$]
	For every $n\in\omega$, let $\Phi_n(\cdot)$ be a formula with one free variable satisfied by exactly one set $X$, then there must be a $k \in\omega$ and two sets $X,Y$ such that 
	\[
		\Phi_k (X) \land \Phi_{k+1} (Y) \land \neg(\mathcal{P}(Y) <_c X)
	\]
\end{theorem}



%Voy agregando bibliografía, con el Kunen y el Jech por si llegan a hacer falta
\bibliographystyle{plain}
\bibliography{bibliografia-hartogs}
\printindex
