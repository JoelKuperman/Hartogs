\section{Problem}
\begin{definition}\label{aleph}
	Let $X$ be any set,
	\[
	\text{WO}(X):=\{\langle A,R\rangle\ |A\subset X \ \text{is well-ordered by}\ R\} := \text{W}, 
	\]
	and $\aleph(X):= \text{W}/{\sim_X}$ where $\langle A,R\rangle \sim_X \langle B, R'\rangle \Longleftrightarrow \langle A, R\rangle =_o \langle B, R'\rangle$ and $=_o$ denotes initial similarity (Moschovakis, def. 7.27 \cite{Moscho}).
	Given $\aleph(X)$, and $A,B \in\aleph(X)$ we shall say $A \leq_{\aleph(X)} B$ iff for any (all) $U\in A$ and $V\in B$ it holds that $U\leq_o V$, where this last relation means $U$ is isomorphic to an initial segment of $V$.
\end{definition}

\begin{lemma}\label{aleph2}
	For all $X$, $\aleph(X)$ has the following properties:
	\begin{enumerate}[(i)]
		\item $\aleph(X)\nleq_c X$
		\item $(\aleph(X),\leq_{\aleph(X)})$ is a well order and is not equinumerous with any of its initial segments.
	\end{enumerate}
\end{lemma}

\begin{theorem}
	For any set $X$, $\aleph(X)\leq_c \mathcal{P}^{4}(X)$.
	%$\aleph(X)<_c \mathcal{P}^4(X)$.
\end{theorem}

\begin{proof}
	We will show that there's an injective function from $\aleph(X)$ into $\mathcal{P}^4(X)$. Let $f:WO(X)\rightarrow \mathcal{P}^3(X)$ defined by $f(\langle A,R\rangle) = \{\{\{a\},\{a,b\}\}| a\leq_R b\}$. Note that $f$ is injective. Now, we can define an injective function $\Tilde{f}: \mathcal{P} (WO(X)) \rightarrow \mathcal{P}^4(X)$ given by $\Tilde{f}(S) = \{f(s) | s\in S\}$. This shows that $\mathcal{P} (WO(X)) \leq_c \mathcal{P}^4(X)$. Hence, since $\aleph (X) \subset \mathcal{P} (WO(X))$, we obtained what we wanted to show.
\end{proof}

\begin{theorem}\label{partes3}
	For any set $X$, $\aleph(X)\leq_c \mathcal{P}^3(X)$.
	%$\aleph(X)<_c \mathcal{P}^3(X)$.
\end{theorem}

This is the best bound we can obtain in ZF: it is consistent with ZF that the stronger inequality $\aleph(X) <_c \mathcal{P}^2(X)$ fails. ¿CITA DE ESTO? NO ME ACUERDO SI ERA MÁS O MENOS DIRECTO.
% Cambio BST por ZF porque todavía no lo definimos

% ¿Es consistente en BST que no valga con $\mathcal(P)^{2}$ para algun X?

\begin{proof}	
	Let $f:WO(X)\rightarrow \mathcal{P}^2(X)$ defined by $f(\langle A,\mathrel{R}\rangle) = \{\{b\mid b \mathrel{R} a\} \mid a\in A\}$. $f$ is clearly injective and an argument similar as before yields what we wanted to show.
\end{proof}

\begin{question}
	Does ZF prove the strict inequality $\forall \ X \ \aleph(X) <_c \mathcal{P}^3(X)$?
\end{question}

The full strength of ZFC proves $\forall X \ \aleph(X) <_c \mathcal{P}^{2}(X)$, moreover, the statement $\forall X \ \aleph(X) =_c \mathcal{P}(X)$ (GCH) and its negation are both consistent with ZFC. The above remark naturally leads to the question "what can we say about the consistency of $\text{ZFC} + \forall X \ \aleph(X) <_c \mathcal{P}(X)$?".

Matthew Foreman and W. Hugh Woodin proved in 1991 \cite{GCH_Fail} that, assuming there is a supercompact cardinal with infinitely many inaccessible cardinals above it, it is consistent with ZFC for the relation $\aleph(X) <_c \mathcal{P}(X)$ to hold for every $X$. 

This result by Foreman and Woodin motivates the question "is this large cardinal assumption necessary?", or equivalently:

\begin{question}
	Does $\text{ZFC} + \forall X \ \aleph(X) <_c \mathcal{P}(X)$ 
	prove the existence of a supercompact cardinal?
\end{question}

\begin{theorem}[ZFC]
	There is no sequence $\{X_n\}_{n\in\omega}$ such that for all $n\in\omega$ $X_{n+1} <_c X_n$.
\end{theorem}
\begin{proof}
	Assume by contradiction that there is such a sequence, and let us take $\{\alpha_n\}_{n\in\omega}$ such that each $\alpha_n$ is an ordinal with the same cardinality as $X_n$. We then have an infinite sequence of ordinals of decreasing cardinality, which leads us to a contradiction.
\end{proof}

This theorem is called the Non Decreasing Sequence principle (NDS) and it is an open problem whether $\text{ZF$+$NDS}{\implies}\text{ZFC}$.
It is implied by AC, and in its absence it is consistent for it to fail in any way: given a partially ordered set $\langle P,< \rangle$ it is consistent with ZF for there to be a family of sets $\mathcal{F}$ such that $\langle \mathcal{F},<_c\rangle$ is isomorphic to $\langle P,< \rangle$ (Jech, Theorem 11.1 \cite{Jech_AC}).

APARENTEMENTE ZFA$+$NDS$+\neg$AC ES CONSISTENTE (BUSCAR CITA), VER CÓMO ES EL MODELO QUE CONSTRUYEN DE ESO.
PARECE QUE \cite{NDS} PRUEBA ESTO, LEER BIEN QUÉ PRUEBAN EXACTAMENTE.

We shall use $\estf$ (Elementary Set Theory) to mean the axioms of Extensionality, Comprehension, Pairing, Union and Powerset. 
Without Replacement or Choice it is consistent for several common definitions of infinite set to not be equivalent between each other, (see Dedekind finite sets, amorphous sets, etc.) which motivates the following weaker version of the Axiom of Infinity:

\begin{WeakInf}
	There is an infinite set.
	%There is a well ordering $\langle A , < \rangle$ such that for all $x\in A$, $x$ has an immediate successor.
\end{WeakInf}

We shall use $\esti$ to denote $\estf$ with the Weak Axiom of Infinity and AI to mean the Axiom of Infinity in its usual formulation.
Note that it is consistent with ZC$-$AI for an infinite well order to exist and for the Axiom of Infinity to fail (\cite[Exercise II.4.21]{Kunen} gives instructions on how to build such a model), in particular $\esti+\neg \text{AI}$ is consistent relative to ZFC.

Given that we shall be working in an axiomatic system too weak to prove the existence of the ordinal $\omega$, we would like to have a first order sentence that essentially states ``the order $\langle X , < \rangle$ would be isomorphic to $\omega$ if the latter existed''.

\begin{definition}
	Whenever $\langle A,< \rangle$ is a linear ordering, we will use the following abbrevations:
	
	\begin{enumerate}[(i)]
		\item $\text{First}(x) \equiv \forall y \in A \neg(y < x)$
		\item $\text{Succ}(y,x) \equiv y<x \land \forall z\in A (z < x \implies (z=y \lor z < y))$
	\end{enumerate}

	Whenever $x$ and $y$ satisfy $\text{Succ}(y,x)$, we shall denote $x = y+1$. For good measure: $y+2 := (y+1)+1$, $y+3 := (y+2)+1$ and $y+4 := (y+3)+1$.
\end{definition}

\begin{definition}
	\label{omega_type}
	Let $\langle A,< \rangle$ be a non-empty well ordering, we will denote $\type(A,<) = \omega$ iff $\langle A,< \rangle$ satisfies the following sentence:
	\[
		\forall x\in A \ ((\text{First}(x) \lor \exists y\in A \ (\text{Succ}(y,x))) \land \exists z\in A \ (\text{Succ}(x,z)))
	\]
\end{definition}

A useful definition to work with recursion in the absence of any instance of Replacement is the following.

\begin{definition}\cite[Definition 5.1]{Moscho}
	A \emph{Peano System} is any structure $(\mathbb{N},0,S)$ which satisfies the following properties:
	\begin{enumerate}[(1)]
		\item $\mathbb{N}$ is a set and $0\in\mathbb{N}$.
		\item $S$ is a function on $\mathbb{N}$, $S: \mathbb{N}\rightarrow\mathbb{N}$.
		\item $S$ is an injection.
		\item For each $n\in\mathbb{N}$, $S(n) \neq 0$.
		\item For each $X\subseteq\mathbb{N}$,
		\[
			(0\in X \land (\forall n\in \mathbb{N} \ (n\in X \implies S(n)\in X))) \implies X = \mathbb{N}.
		\]
	\end{enumerate}
\end{definition}

\begin{lemma}
	Assume $\langle A,< \rangle$ is a well order such that $\type(A,<) = \omega$, let $a_0$ be its first element. Then the following are provable in \emph{$\esti$}:
	\begin{enumerate}[(i)]
		\item There is a function $S_A : A \rightarrow A$ such that $\forall x\in A$ $\emph{\text{Succ}}(x,S_A (x))$.
		\item The structure $(A,a_0,S_A)$ is a Peano system.
	\end{enumerate}
\end{lemma}
\begin{proof}
	Item \emph{(i)} requires only to use Comprehension on the set $A\times A$: we simply take $S_A$ as the set $\{(x,y)\in A\times A: \text{Succ}(x,y)\}$. Let us now prove item \emph{(ii)}. Properties (1) through (4) of the definition of a Peano system are an immediate consequence of the hypothesis on $\langle A,< \rangle$. Let $X\subseteq A$ be a set such that 
	\[
		a_0\in X \land (\forall n\in A \ (n\in X \implies S_A(n)\in X)).
	\]
	Assume $X\neq A$ and let $a'$ be the first element of $A\setminus X$. Using the hypothesis $\type(A,<) = \omega$ we have that $a'$ is the first element of $A$ or there is some $b\in A$ such that $a' = S_A (b)$. In the first case we would have $a' = a_0$, which contradicts the assumption on $X$. So there must be some $b\in A$ such that $S_A (b) = a'$. But $a'$ is the first element of $A\setminus X$ and $b<a'$, which implies $b\in X$. Finally $S_A(b) \in X$, and the contradiction arose from assuming $A\setminus X$ was non-empty.
\end{proof}

\begin{lemma}[$\esti$]
	Let $\langle A,< \rangle$ and $\langle B,\sqsubset \rangle$ be two well orders such that $\type(A,<) = \omega$ and $\type(B,\sqsubset) = \omega$, then $\langle A,< \rangle$ and $\langle B,\sqsubset \rangle$ are order isomorphic.
\end{lemma}
\begin{proof}
	Let's denote by $a_0$ and $b_0$ the $<$-minimal element of $\langle A, <\rangle$ and the $\sqsubset$-minimal element of $\langle B, \sqsubset \rangle$ respectively. Now, we define $f$ recursively by $f(a_0)=b_0$, $f(x+1)=f(x)+1$. By the previous lemma and the Recursion Theorem \cite[Theorem 5.6]{Moscho}, there is exactly one function that satisfies the previous equalities. We claim that $f$ is an order isomorphism: 
	\begin{itemize}
		\item $x<y \implies f(x)\sqsubset f(y)$. Let $x\in A$ and suppose there exists $y>x$ such that $f(y)\sqsubset f(x)$. Take $y$ as the $<$-minimal element with this property. It cannot be the case that $y=a_0$ nor that $y=x+1$ (since this would imply $f(y)=f(x)+1)$. Then, there must exist a $z\neq x$ such that $y=z+1)$ and since $x<y$, this implies $x<z$. By hypothesis about $y$, this yields $f(x)\sqsubset f(z)$, but then $f(x)\sqsubset f(z)+1 = f(y)$, which is a contradiction. Note that this condition also proves that $f$ is injective.
		\item $f$ is surjective. Indeed, supose there's an $y\in B$ such that $y\notin f(A)$. Take $y$ as the $\sqsubset$-minimal element with this property. Of course, $y\neq b_0$, so there must exist a $z\in B$ such that $y=z+1$. By hypothesis, we get that there exists an $x$ such that $f(x)=z$. But then, $f(x+1)=z+1=y$, a contradiction.
		\item The fact that the inverse of $f$ is order-preserving follows by the same argument used to show that $f$ itself is order-preserving.
	\end{itemize} 
\end{proof}

\begin{lemma}[$\estf+\text{AI}$]
	$\langle \omega,\in \rangle$ satisfies the sentences from Definition \ref{omega_type}. \qed
\end{lemma}

\begin{corollary}[$\estf+\text{AI}$]
	Our notation  $\type(A,<) = \omega$ coincides with its standard form stating that $\omega$ is an ordinal and $\langle A,< \rangle$ is order isomorphic to it. \qed
\end{corollary}

\begin{lemma}
	If $\langle X,< \rangle$ is an infinite well order such that $\neg \type(X,<) = \omega$, there is some $x\in X$ such that the restriction of $<$ to $x{\downarrow} \times x{\downarrow}$ has type $\omega$.
\end{lemma}
\begin{proof}
	Since $\langle X,< \rangle$ does not have type $\omega$, there must be some $x\in X$ such that
	\[
		\neg(\text{First}(x) \lor \exists y\in X \ (\text{Succ}(y,x)) \lor \neg \exists z\in X \ (\text{Succ}(x,z)).
	\]
	We shall see that in both cases $x{\downarrow}$ is infinite. Let us assume 
	\[
		\neg(\text{First}(x) \lor \exists y\in X \ (\text{Succ}(y,x)),
	\]
	then if $x{\downarrow}$ is non empty and it must be infinite, indeed: if it was finite, it would have a greatest element $y$, which would imply $\text{Succ}(y,x)$.
	
	On the other hand, if we assume
	\[
		\neg \exists z\in X \ (\text{Succ}(x,z)),
	\]
	then $X = x{\downarrow} \cup \{x\}$. Given that $X$ is infinite by assumption, it follows that $x{\downarrow}$ must be infinite.
	
	Finally, let us take $Y = \{x\in X : x{\downarrow} \text{ is infinite}\}$. This set is non empty by the previous argument, so it has a least element $y$. We will prove that $\type(y\downarrow,<) = \omega$.
	Recall that we need to show:
	\[
		\forall a\in y{\downarrow} \ ((\text{First}(a) \lor \exists b\in y{\downarrow} \ (\text{Succ}(b,a))) \land \exists c\in y{\downarrow} \ (\text{Succ}(a,c)))
	\]
	Let $a\in y{\downarrow}$, since $y$ is the first element whose set of predecessors is infinite, $a{\downarrow}$ is necessarily finite. In particular it is either empty, in which case First$(a)$, or it has a greatest element $b$ which satisfies $b\in y{\downarrow}$ and Succ$(b,a)$. All that remains to verify is the existence of some $c\in y{\downarrow}$ such that Succ$(a,c)$.
	First let us note that $a<y$ and $\neg\text{Succ}(a,y)$, so the set $\{z\in y{\downarrow} : a<z\}$ is non empty. Taking its first element we obtain a $c$ with the desired property.
\end{proof}

\begin{prop}[$\esti$]
	There is a well ordering $\langle A,< \rangle$ such that 
	$\type(A,<) = \omega$.
\end{prop}
\begin{proof}
	Let $X$ be any infinite set, of course if there was a well order on it we could simply restrict it to an appropriate subset and get one of type $\omega$. In the absence of Choice, we may not be so fortunate. Let us take $A = \aleph(X)$. Using the previous lemma, all we need to do is to show that $A$ is not finite.
	But this is trivial by Lemma \ref{aleph2}, if it was finite, then we would have $\aleph(X) <_c X$.
\end{proof}

\begin{theorem}[$\esti$]
	\label{hartogs_set}
	Let $\langle A,< \rangle$ be a well order such that $\type(A,<) = \omega$, then there is no sequence $\{X_n\}_{n\in A}$ such that for all $n\in A$ ($\mathcal{P}(X_{n+1}) <_c X_n$).
\end{theorem}

Before moving on to the proof, let us note that Definition \ref{aleph} does not rely on any axiom outside of $\estf$, with the caveat that without Replacement, $\aleph(X)$ cannot be proven to be isomorphic to an ordinal.

\begin{proof}
	Assume there is such a sequence, 
	$\mathcal{S} := \{X_n\}_{n\in A}$,
	then $\estf$ proves that $\ran(\mathcal{S})$ exists (\cite[Def. I.6.6]{Kunen}).
	Moreover, for all $X\in\ran(\mathcal{S})$: 
	\[
	\aleph(X) \subset \mathcal{P}(\WO(X)) \subseteq \mathcal{P}(\WO(\union \ran(\mathcal{S}))),
	\]
	where both inclusions follow from Definition \ref{aleph}.
	
	\noindent Thus, for all $X\in\ran(\mathcal{S})$, we have $\aleph(X)\in\mathcal{P}(\mathcal{P}(\WO(\union \ran(\mathcal{S}))))$. Hence, $\esti$ proves the existence of the sequence $\{\aleph(X_n)\}_{n\in A}$, which would of course be trivial using Replacement.
	
	It is straightforward to prove that for each $n\in A$, 
	%$\mathcal{P}^{3}(X_{n+3}) <_c X_n$, 
	$\mathcal{P}^{4}(X_{n+4}) <_c X_n$, 
	indeed, applying the hypothesis on $\mathcal{S}$ yields:
	
%	\begin{align*}
%		&\mathcal{P}(X_{n+3}) \hspace{4.5 pt} <_c X_{n+2}
%		\\ \implies &\mathcal{P}^2(X_{n+3}) \leq_c \hspace{4.5 pt} \mathcal{P}(X_{n+2}) <_c X_{n+1}
%		\\ \implies &\mathcal{P}^3(X_{n+3}) \leq_c \mathcal{P}^{2}(X_{n+2}) \leq_c \mathcal{P}(X_{n+1}) <_c X_n.
%	\end{align*}
	
	\begin{align*}
		&\mathcal{P}(X_{n+4}) \hspace{4.5 pt} <_c X_{n+3}
		\\ \implies &\mathcal{P}^2(X_{n+4}) \leq_c %\hspace{4.5 pt}
		 \mathcal{P}(X_{n+3}) \hspace{4.5 pt} <_c X_{n+2}
		\\ \implies &\mathcal{P}^3(X_{n+4}) \leq_c \mathcal{P}^{2}(X_{n+3}) \leq_c \mathcal{P}(X_{n+2}) \hspace{4.5 pt} <_c X_{n+1}
		\\ \implies &\mathcal{P}^4(X_{n+4}) \leq_c \mathcal{P}^3(X_{n+3}) \leq_c \mathcal{P}^2(X_{n+2}) \leq_c \mathcal{P}^1(X_{n+1}) <_c X_n.
	\end{align*}
	As a consequence,
	%$\aleph(X_{n+3}) <_c \mathcal{P}^{3}(X_{n+3}) <_c X_n$,
	$\aleph(X_{n+4}) <_c \mathcal{P}^{4}(X_{n+4}) <_c X_n$,
	where the first inequality comes from applying Theorem \ref{partes3}.
	
	\noindent Let us now show that $\aleph(X_{n+4}) <_c \aleph(X_n)$ for all $n\in A$. Well ordered sets are pairwise comparable with the relation $\leq_c$, in particular: $\aleph(X_{n+4}) <_c \aleph(X_n)$ or $\aleph(X_n) \leq_c \aleph(X_{n+4})$. The latter statement cannot hold, since it would imply $\aleph(X_n) <_c X_n$, which contradicts Lemma \ref{aleph2}.
	
	Finally, $\{\aleph(X_n)\}_{n\in A}$
	%$\{\aleph(X_{3n})\}_{n\in\omega}$ 
	is a set of well ordered sets that is not well founded with the $<_c$ relation, which cannot exist under $\estf$. Indeed, 
	it is non empty and it has no least element: for any $\aleph(X_n)$, we have $\aleph(X_{n+4}) <_c \aleph(X_n)$.
	
\end{proof}

\begin{notation}
	Given that without the Axiom of Infinity $\omega$ could be a proper class, when we say $n\in\omega$ we shall really mean $\forall x\in n \ (x = \emptyset \lor \exists y \in n \ x = S(y))$, where $S(a) = a \cup \{a\}$.
\end{notation}

The following theorem is equivalent to the previous one when using Replacement but strictly stronger when working within $\estf$, it essentially shows that any formula defining a sequence like the one from %(¿LE PONEMOS NOMBRE A ESTA PROPIEDAD?)
Theorem \ref{hartogs_set} as a proper class of ordered pairs must be inconsistent with $\estf$.

\begin{theorem}[$\estf$]
	\label{hartogs_class}
	Let $\Phi(\cdot,\cdot)$ be a formula with two free variables, then $\Phi$ cannot satisfy both of the following properties:
	\begin{enumerate}[(i)]
		\item \label{nat_define} For each $n\in\omega$ there is exactly one set $X$ that satisfies $\Phi(n,X)$.
		\item \label{dec_seq} For each $n\in\omega$ the following holds: 
		\[
			(\Phi(n,X) \land \Phi(n+1,Y)) \implies \mathcal{P}(Y) <_c X.
		\] 
	\end{enumerate}
\end{theorem}

Before giving a proof for Theorem \ref{hartogs_class} we note the following result from \cite[Corollary 7.33]{Moscho}, which can be proven in $\estf$.

\begin{prop}\label{class_wo}
	Every non-empty class $\mathcal{C}$ of well ordered sets has a $\leq_o$-least member, i.e., for some $U_0 \in \mathcal{C}$ and all $U\in\mathcal{C}$, $U_0$ is isomorphic to an initial segment of $U$.
\end{prop}

\begin{proof}[Proof of Theorem \ref{hartogs_class}]
	Let $\Phi$ be a formula with two free variables and assume it satisfies (\ref{nat_define}) and (\ref{dec_seq}), we shall take the following class:
	\[
		 \mathcal{C} := \{X: \exists Y ((\exists n\in\omega \ \Phi(n,Y)) \land X = \aleph(Y))\}.
	\]
	Using (\ref{nat_define}), Definition \ref{aleph} and Lemma \ref{aleph2}, $\mathcal{C}$ is a non-empty class of well ordered sets. By Proposition \ref{class_wo}, there must be a $\leq_o$-least member of $\mathcal{C}$.
	Let $X_0$ be such a member of $\mathcal{C}$, then there must be some $Y_0$ and $n_0 \in\omega$ such that $X_0 = \aleph(Y_0) \land \Phi(n_0,Y_0)$.
	The assumption (\ref{nat_define}) implies there are $Y_1 , Y_2, Y_3, Y_4$ such that $\Phi(n_0 + i , Y_i)$ for $i\in\{1,2,3,4\}$.
	Applying (\ref{dec_seq}), we get:	

%	\begin{align*}
%		&\mathcal{P}(Y_3) \hspace{4.5 pt} <_c Y_2
%		\\ \implies &\mathcal{P}^2(Y_3) \leq_c \hspace{4.5 pt} \mathcal{P}(Y_2) <_c Y_1
%		\\ \implies &\mathcal{P}^3(Y_3) \leq_c \mathcal{P}^{2}(Y_2) \leq_c \mathcal{P}(Y_1) <_c Y_0.
%	\end{align*}
	
	\begin{align*}
		&\mathcal{P}(Y_4) \hspace{4.5 pt} <_c Y_3
		\\ \implies &\mathcal{P}^2(Y_4) \leq_c \mathcal{P}(Y_3) \hspace{4.5 pt} <_c Y_2
		\\ \implies &\mathcal{P}^3(Y_4) \leq_c \mathcal{P}^{2}(Y_3) \leq_c \mathcal{P}(Y_2) \hspace{4.5 pt} <_c Y_1
		\\ \implies &\mathcal{P}^4(Y_4) \leq_c \mathcal{P}^3(Y_3) \leq_c \mathcal{P}^2(Y_2) \leq_c \mathcal{P}^1(Y_1) <_c Y_0
	\end{align*}
	
	Recall from Theorem \ref{partes3} that $\aleph(Y_4) \leq_c \mathcal{P}^{3}(Y_4)$, and using the inequality above, we obtain $\aleph(Y_4) <_c Y_0$, which implies $\aleph(Y_4) <_c \aleph(Y_0) = X_0$. But $\aleph(Y_4)\in\mathcal{C}$, which contradicts $X_0$ being $\leq_o$ minimal. 
\end{proof}

OTRA DUDA, ¿EN $\estf$ VALE EL SIGUIENTE TEOREMA? ES SIQUIERA ENUNCIABLE?

We would like to prove the following stronger version of Theorem \ref{hartogs_class}.

\begin{theorem}
	\label{hartogs_inf_classes}
	For every $n\in\mathbb{N}$, let $\Phi_n(\cdot)$ be a formula with one free variable such that for each $n\in\mathbb{N}$, $\estf \vdash \exists! X \ \Phi_n (X)$, then there must be a $k \in \mathbb{N}$ such that 
	\[
		\estf \vdash \exists X,Y ,\, \Phi_k (X) \land \Phi_{k+1} (Y) \land \neg(\mathcal{P}(Y) <_c X)
	\]
\end{theorem}

\begin{notation}
	We shall take WO($A,<$), or WO($A$) when $<$ is clear from context, to be an abbreviation of a first order logic formula stating ``$\langle A,<\rangle$ is a well-order''.
\end{notation}

We would like to show the equivalence of the previous theorem to the following result.

\begin{lemma}
	\label{easier_hartogs_class}
	For every $n\in\mathbb{N}$, let $\Phi_n(\cdot)$ be a formula with one free variable such that for each $n\in\mathbb{N}$, 
	\[
		\estf \vdash \exists! \langle X, < \rangle \ \Phi_n ( \langle A,<\rangle) \land 
		\emph{WO}(\langle X, < \rangle)
	\]
	then there must be a $k \in \mathbb{N}$ such that 
	\[
		\estf \vdash \exists A,B, \, \Phi_k (A) \land \Phi_{k+1} (B) \land A \leq_o B
	\]
\end{lemma}

We state a weaker version of Theorem \ref{hartogs_inf_classes}.

\begin{lemma}
	\label{easier_inf_classes}
	For every $n\in\mathbb{N}$, let $\Phi_n(\cdot)$ be a formula with one free variable such that for each $n\in\mathbb{N}$, $\estf \vdash \exists! X \ \Phi_n (X)$, then there must be a $k \in \mathbb{N}$ such that 
	\[
		\estf \vdash \exists X,Y ,\, \Phi_k (X) \land \Phi_{k+1} (Y) \land \neg(\mathcal{P}^{4}(Y) <_c X)
	\]
\end{lemma}

\begin{theorem}
	Lemma \ref{easier_hartogs_class} implies Lemma \ref{easier_inf_classes}.
\end{theorem}

\begin{proof}
	Assume Lemma \ref{easier_hartogs_class} and let $\Phi_n$ be a scheme of formulas with one free variable such that for each $n\in\mathbb{N}$
	\[
		\estf \vdash \exists! X \ \Phi_n (X)
	\]
	Let $\Psi_n (A)$ be the formula $\exists X (\Phi_n (X) \land A = \aleph(X))$.
	Given that for each $n\in\mathbb{N}$, $\estf \vdash \exists! X \ \Phi_n (X)$, we also have 
	\begin{equation}\tag{1}
		\label{psi_unique}
		\estf \vdash \exists! A \ \Psi_n (A)	
	\end{equation}
	So we can apply Lemma \ref{easier_hartogs_class} with the formulas $\Psi_n$. Let $k$ be a natural number such that 
	\begin{equation}\tag{2}
		\label{eq_wo}
		\estf \vdash \exists A,B, \, \Psi_k (A) \land \Psi_{k+1} (B) \land A \leq_o B
	\end{equation}
	We give a proof by contradiction within $\estf$. Let $X$ and $Y$ be the unique sets such that $\Phi_k (X)$ and $\Phi_{k+1} (Y)$. Let $A = \aleph(X)$ and $B = \aleph(Y)$.
	Note that $\Psi_k (A)$ and $\Psi_{k+1} (B)$. By (\ref{psi_unique}) and (\ref{eq_wo}), we have $A \leq_o B$.

	Assume $\mathcal{P}^{4}(Y) <_c X$, applying Theorem \ref{partes3} yields
	\begin{equation}\tag{3}
		\label{eq_order}
		\aleph(Y) <_c \mathcal{P}^{4} (Y) <_c X. 
	\end{equation}
	Given that all well orders are pairwise comparable with the $\leq_c$ relation, it must be the case that $\aleph(Y) <_c \aleph(X)$ or $\aleph(X) \leq_c \aleph(Y)$.
	But $\aleph(X) \leq_c \aleph(Y)$, together with (\ref{eq_order}), would imply $\aleph(X) <_c X$, which contradicts Lemma \ref{aleph2}.
	
	So the inequality $\aleph(Y) <_c \aleph(X)$  must hold. Recall that $\aleph(Y) = B$ and $\aleph(X) = A$, so the previous inequality means $B <_c A$, which contradicts (\ref{eq_wo}).
	
	We have thus derived a contradiction from assuming $\mathcal{P}^{4}(Y) <_c X$, hence
	\[
		\estf \vdash \exists X,Y ,\, \Phi_k (X) \land \Phi_{k+1} (Y) \land \neg(\mathcal{P}^{4}(Y) <_c X)
	\]
\end{proof}

\begin{question}
	Does Lemma \ref{easier_inf_classes} imply Lemma \ref{easier_hartogs_class}?
\end{question}
%\begin{proof}
%	IGNORAR ESTA PRUEBA POR AHORA!!!
%	By contradiction, let us assume Lemma \ref{easier_hartogs_class} does not hold. Then there is a scheme of formulae indexed by $n\in\mathbb{N}$, $\Phi_n$ with one free variable such that for every $n\in\mathbb{N}$
%	\[
%		\estf \vdash \exists! \langle X, < \rangle \ \Phi_n ( \langle X,<\rangle) \land 
%		\emph{WO}(\langle X, < \rangle)
%	\]
%	And for every $k\in\mathbb{N}$ the following must hold
%	\[			
%		\estf \nvdash \exists X,Y, \, \Phi_k (X) \land \Phi_{k+1} (Y) \land X \leq_o Y
%	\]	
%	Using Completeness and Soundness as in the previous proof, there must be for each $k\in\mathbb{N}$ a model $\mathcal{A}_k$ of $\estf$ such that
%	\[
%		\mathcal{A}_k \vDash \forall X,Y, \, (\Phi_k (X) \land \Phi_{k+1} (Y)) \implies X >_o Y
%	\]	
	%Let $\Psi_n (X) = \exists R (R \subseteq {X\times X} \land \Phi_k(\langle X, R\rangle) \land \text{WO}(\langle X, R\rangle))$.
	%Applying Lemma \ref{easier_inf_classes}, there must be a $k\in\mathbb{N}$ such that 
	%\[
	%	\estf \vdash \exists X,Y ,\, \Psi_k (X) \land \Psi_{k+1} (Y) \land \neg(\mathcal{P}^{3}(Y) <_c X)
	%\]
	%Using Soundness again:
	%\[
	%	\mathcal{A}_k \vDash \exists X,Y ,\, \Psi_k (X) \land \Psi_{k+1} (Y) \land \neg(\mathcal{P}^{3}(Y) <_c X)
	%\]
	%Let $X$ and $Y$ be the only sets in $\mathcal{A}_k$ such that 
	%\[
	%	\Psi_k (X) \land \Psi_{k+1} (Y) \land \neg(\mathcal{P}^{3}(Y) <_c X)
	%\]
	%By definition of $\Psi_k$, there are relations $R$ and $S$ that well order $X$ and $Y$.
	%Moreover, $\langle X,R \rangle >_o \langle Y,S\rangle$. This implies $Y <_c X$
%\end{proof}

We would like to have a unified way of referring to well orderings like $\aleph_1$ or $\aleph_\omega$ in the absence of Replacement. We cannot even prove $\omega+\omega$ exists in ZC, much less the aforementioned cardinals, so we will introduce a notion analogous to Definition \ref{omega_type}, which was a way of talking about well orderings that were ``essentially $\omega$''.

\begin{definition}
	Let $\langle A,< \rangle$ be a well ordering.
	\begin{enumerate}[(i)]
		\item Let $x\in A$, then $x{\downarrow} := \{y\in A: y<x\}$ and $<_x = \{(a,b) \in x{\downarrow}\times x{\downarrow} : a<b\}$.
		\item $\langle A,< \rangle$ is an $\estf$ \emph{cardinal} iff for all $x\in A$, $x{\downarrow} <_c A$.
		\item If $\langle A,< \rangle$ is an infinite $\estf$ cardinal, we define its length as $\mathcal{L}(A,<) = \mathcal{L}(A) = \{x\in A: (x{\downarrow},<_x) \text{ is an infinite $\estf$ cardinal}\}$. We give $\mathcal{L}(A)$ the inherited order from $A$.
	\end{enumerate}
\end{definition}

Note that under Replacement, we can take the length of $\aleph_{\alpha}$ and it is a well order isomorphic to $\alpha$. The above definition simply codifies the behavior of the $\alpha$ subscript in theories where we cannot produce ordinals larger than $\omega+\omega$.
The following three lemmas show that the length of EST cardinals has the properties we would expect when working with cardinals in  theories with Replacement. A notable exception is that $\esti$ can only prove the existence of EST cardinals of finite length.

\begin{lemma}
	\label{weak_aleph}
	Let $A$ and $B$ be two infinite $\estf$ cardinals, then 
	$A <_c B$ iff $\mathcal{L}(A) <_o \mathcal{L}(B)$.
\end{lemma}

\begin{proof}
	Recall that $A \leq_o B$ iff $A$ is isomorphic to an initial segment of $B$ and $A =_o B$ iff $A$ is isomorphic to $B$. Naturally, $A <_o B$ iff $A \leq_o B$ and $A \neq_o B$. 
	%Let us start by proving $\emph{(i)}$.
	\begin{itemize}
		\item[($\Rightarrow$)] Assume $A <_c B$, since they are both well ordered, we have $A <_o B$. Let $f: A \mapsto B$ be an order isomorphism onto an initial segment of $B$. To show that $f{\upharpoonright}\mathcal{L}(A)$ is an order isomorphism onto an initial segment of $\mathcal{L}(B)$, we need only prove that its range is contained in $\mathcal{L}(B)$.
		Working towards a proof by contradiction: assume $\ran(f{\upharpoonright}\mathcal{L}(A)) \not\subseteq \mathcal{L}(B)$ and take the first $x\in\mathcal{L}(A)$ such that $f(x)\notin\mathcal{L}(B)$. Then $f(x){\downarrow}$ is not an EST cardinal, which means there is some $y\in x{\downarrow}$ such that 
		\[
			\hspace{-0.4cm} f(y){\downarrow} =_c f(x){\downarrow}.
		\] 
		Note that $f$ being an order isomorphism onto its image implies 
		\begin{align*}
			x{\downarrow} &=_c f(x){\downarrow}
			\\ y{\downarrow} &=_c f(y){\downarrow}.
		\end{align*}
		 And so $y{\downarrow} = x{\downarrow}$. But $x$ is an EST cardinal, so $y{\downarrow} <_c x{\downarrow}$, which contradicts the previous equalities.
		\item[($\Leftarrow$)] Recall that any two well orders are comparable with the $<_o$ relation, so either $A <_o B$ or $B \leq_o A$. By contradiction: let us assume $\mathcal{L}(A) <_o \mathcal{L}(B)$ and $B \leq_o A$. Any order isomorphism from $B$ onto an initial segment of $A$ can be restricted to $\mathcal{L}(B)$ and has its range contained in $\mathcal{L}(A)$. So we have $\mathcal{L}(A) <_o \mathcal{L}(B)$ and $\mathcal{L}(B) \leq_o \mathcal{L}(A)$. In particular:
		\[
			\mathcal{L}(A) \leq_o \mathcal{L}(B) \land
			\mathcal{L}(B) \leq_o \mathcal{L}(A).
		\]
		Finally, this implies $\mathcal{L}(A) =_o \mathcal{L}(B)$ (\cite[Proposition 7.29]{Moscho}), which contradicts the assumption $\mathcal{L}(A) <_o \mathcal{L}(B)$. \qedhere
	\end{itemize}
\end{proof}

\begin{lemma}
	Let $A$ and $B$ be two infinite $\estf$ cardinals, then $A =_c B$ iff $\mathcal{L}(A) =_o \mathcal{L}(B)$.
\end{lemma}
\begin{proof}
	Let us note that our problem is equivalent to showing that $A \neq_c B$ iff $\mathcal{L}(A) \neq_o \mathcal{L}(B)$.
	Via the trichotomy of the $<_o$ relation for well orders, we obtain
	\begin{align*}
		A \neq_c B &\iff A\neq_c B \land (A \leq_o B \lor B \leq_o A)
		\\ 		   &\iff (A\neq_c B \land A \leq_o B) \lor (A\neq_c B \land B \leq_o A)
		\\		   &\iff A <_o B \lor B <_o A
		\\ \text{(By Lemma \ref{weak_aleph})}		   &\iff \mathcal{L}(A) <_o \mathcal{L}(B) \lor \mathcal{L}(B) <_o \mathcal{L}(A)
		\\ 		   &\iff \mathcal{L}(A) \neq_o \mathcal{L}(B).
	\end{align*}
\end{proof}

\begin{lemma}
	For any set $X$, $\aleph(X)$ is an $\estf$ cardinal. Moreover, if $X$ is not finite, $\aleph(X)$ is an infinite $\estf$ cardinal.
\end{lemma}
\begin{proof}
	Immediate from Lemma \ref{aleph2}.
\end{proof}

In this manner, any infinite cardinal whose $\mathcal{C}$ has exactly one element behaves like $\aleph_1$, even when the latter does not exist as an ordinal.

%Dejo comentada esta prueba por si había algo rescatable acá

%\begin{proof}
%	\emph{Sketch of a proof of the equivalence between \ref{hartogs_inf_classes} and \ref{easier_hartogs_class}.}
	
%	Given $\Phi_n (\cdot)$, we can take $\Psi_n (X) \equiv \exists Y \ (\Phi_n (Y) \land X = \aleph(Y))$. Using our usual trick of taking the sequence of the alephs, we get a sequence of well ordered sets. Now we apply Lemma \ref{weak_aleph}, which gives us one of the directions of the equivalence, the other should be trivial.
%\end{proof}

%Voy agregando bibliografía, con el Kunen y el Jech por si llegan a hacer falta
\bibliographystyle{plain}
\bibliography{bibliografia-hartogs}
\printindex
