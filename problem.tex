\section{Problem}
\begin{definition}\label{aleph}
	Let $X$ be any set,
	\[
	\text{WO}(X):=\{\langle A,R\rangle\ |A\subset X \ \text{is well-ordered by}\ R\} := \text{W}, 
	\]
	and $\aleph(X):= \text{W}/{\sim_X}$ where $\langle A,R\rangle \sim_X \langle B, R'\rangle \Longleftrightarrow \langle A, R\rangle =_o \langle B, R'\rangle$ and $=_o$ denotes initial similarity (Moschovakis, def. 7.27 \cite{Moscho}).
	Given $\aleph(X)$, and $A,B \in\aleph(X)$ we shall say $A \leq_{\aleph(X)} B$ iff for any (all) $U\in A$ and $V\in B$ it holds that $U\leq_o V$, where this last relation means $U$ is isomorphic to an initial segment of $V$.
\end{definition}

\begin{lemma}\label{aleph2}
	For all $X$, $\aleph(X)$ has the following properties:
	\begin{enumerate}[(i)]
		\item $\aleph(X)\nleq_c X$
		\item $(\aleph(X),\leq_{\aleph(X)})$ is a well order and is not equinumerous with any of its initial segments.
	\end{enumerate}
\end{lemma}

%\begin{claim}
%   $\aleph(X)$ is a cardinal and $\aleph(X)\nleq_c X$
%\end{claim}

\begin{theorem}
	For any set $X$, $\aleph(X)<_c \mathcal{P}^4(X)$.
\end{theorem}

\begin{proof}
	We will show that there's an injective function from $\aleph(X)$ into $\mathcal{P}^4(X)$. Let $f:WO(X)\rightarrow \mathcal{P}^3(X)$ defined by $f(\langle A,R\rangle) = \{\{\{a\},\{a,b\}\}| a\leq_R b\}$. Note that $f$ is injective. Now, we can define an injective function $\Tilde{f}: \mathcal{P} (WO(X)) \rightarrow \mathcal{P}^4(X)$ given by $\Tilde{f}(S) = \{f(s) | s\in S\}$. This shows that $\mathcal{P} (WO(X)) \leq_c \mathcal{P}^4(X)$. Hence, since $\aleph (X) \subset \mathcal{P} (WO(X))$, we obtained what we wanted to show.
\end{proof}

\begin{theorem}\label{partes3}
	For any set $X$, $\aleph(X)<_c \mathcal{P}^3(X)$.
\end{theorem}

This is the best bound we can obtain in BST: it is consistent with BST that the stronger inequality $\aleph(X)<\mathcal{P}^2(X)$ fails. 
% Ver si es consistente con ZFC que valga $\aleph(X)<_c \mathcal{P}(X)$ (es consistente el igual en todos lados, ver si pueden fallar todas las instancias de GCH).
% GCH puede fallar en todos lados: https://www.jstor.org/stable/2944324
% Ver si se puede probar en ZFC que $\aleph(X)<_c \mathcal{P}^2(X)$. Prueba rapida: $|X| = \kappa$, $\aleph(X) = \kappa^{+} \leq_c 2^{\kappa} <_c 2^{2^{\kappa}} = |\mathcal(P)^{2}(\kappa)|$.
% ¿Es consistente en BST que no valga con $\mathcal(P)^{2}$ para algun X?

\begin{proof}
	%	We will show that there's an injective function from $\aleph(X)$ into $\mathcal{P}^4(X)$. Let $f:WO(X)\rightarrow \mathcal{P}^3(X)$ defined by $f(\langle A,R\rangle) = \{\{\{a\},\{a,b\}\}| a\leq_R b\}$. Note that $f$ is injective. Now, we can define an injective function $\Tilde{f}: \mathcal{P} (WO(X)) \rightarrow \mathcal{P}^4(X)$ given by $\Tilde{f}(S) = \{f(s) | s\in S\}$. This shows that $\mathcal{P} (WO(X)) \leq_c \mathcal{P}^4(X)$. Hence, since $\aleph (X) \subset \mathcal{P} (WO(X))$, we obtained what we wanted to show.
	%\end{proof}
	
	Let $f:WO(X)\rightarrow \mathcal{P}^2(X)$ defined by $f(\langle A,\mathrel{R}\rangle) = \{\{b\mid b \mathrel{R} a\} \mid a\in A\}$. $f$ is clearly injective and an argument similar as before yields what we wanted to show.
\end{proof}

\begin{theorem}[ZFC]
	There is no sequence $\{X_n\}_{n\in\omega}$ such that for all $n\in\omega$ $X_{n+1} <_c X_n$.
\end{theorem}
\begin{proof}
	Assume by contradiction that there is such a sequence, and let us take $\{\alpha_n\}_{n\in\omega}$ such that each $\alpha_n$ is an ordinal with the same cardinality as $X_n$. We then have an infinite sequence of ordinals of decreasing cardinality, which leads us to a contradiction.
\end{proof}

% Ver si en ZF, el teorema anterior es equivalente a AC. AC lo implica, falta ver la vuelta.
% Se llama NDS el teorema anterior.
% Lectura inicial:  https://link.springer.com/article/10.1007/s00153-015-0472-5#:~:text=Abstract,m%E2%80%9D%20%E2%86%9B%20NDS%20in%20ZFA

The previous result relies heavily on Choice, in fact, given any countable partially ordered set $\langle P,< \rangle$ it is consistent with ZF for there to be a family of sets $\mathcal{F}$ such that $\langle \mathcal{F},<_c\rangle$ is isomorphic to $\langle P,< \rangle$.
%CONSEGUIR CITA DE LO ANTERIOR
%También, ver si puede valer el teorema anterior y fallar Elección, posiblemente salga de
%estudiar los modelos en los que falla.

We shall use $\estf$ (Elementary Set Theory) to mean the axioms of Extensionality, Comprehension, Pairing, Union and Powerset. 
Without Replacement or Choice it is consistent for several common definitions of infinite set to not be equivalent between each other, 
(VER BIEN CUÁLES HAY: SE ME OCURREN CONJUNTOS AMORFOS, DEDEKIND FINITOS NO FINITOS, EL AXIOMA DE INFINITO DE ZERMELO) which motivates the following weaker version of the Axiom of Infinity:

\begin{WeakInf}
	There is a well ordering $\langle A , < \rangle$ such that for all $x\in A$, $x$ has an immediate successor.
\end{WeakInf}

Let us use $\esti$ to denote $\estf$ with the Weak Axiom of Infinity.

\begin{question}
	Is it consistent with $\esti$ for $\omega$ to not exist? (SEE ZERMELO'S AXIOM OF INFINITY).
\end{question}

Given that we shall be working in an axiomatic system too weak to prove the existence of the ordinal $\omega$, we would like to have a first order sentence that essentially states "the order $\langle X , < \rangle$ is isomorphic to $\omega$ if the latter exists".

\begin{definition}
	Whenever $\langle A,< \rangle$ is a linear ordering, we will use the following abbrevations:
	
	\begin{enumerate}[(i)]
		\item $\text{First}(x) \equiv \forall y \in A \neg(y < x)$
		\item $\text{Succ}(y,x) \equiv y<x \land \forall z\in A (z < x \implies (z=y \lor z < y))$
	\end{enumerate}

	Whenever $x$ and $y$ satisfy $\text{Succ}(y,x)$, we shall denote $x = y+1$. For good measure: $y+2 := (y+1)+1$, $y+3 := (y+2)+1$ and $y+4 := (y+3)+1$.
\end{definition}

\begin{definition}
	\label{omega_type}
	Let $\langle A,< \rangle$ be a well ordering, we will denote $\type(A,<) = \omega$ iff $\langle A,< \rangle$ satisfies the following sentence:
	\[
		\forall x\in A \ ((\text{First}(x) \lor \exists y\in A \ (\text{Succ}(y,x))) \land \exists z\in A \ (\text{Succ}(x,z)))
	\]
\end{definition}

\begin{lemma}[$\esti$]
	Let $\langle A,< \rangle$ and $\langle B,\sqsubset \rangle$ be two well orders such that $\type(A,<) = \omega$ and $\type(B,\sqsubset) = \omega$, then $\langle A,< \rangle$ and $\langle B,\sqsubset \rangle$ are order isomorphic.
\end{lemma}
\begin{proof}
	¿TRIVIAL?
\end{proof}

\begin{lemma}[$\esti$]
	$\type(\omega,\in) = \omega$
\end{lemma}
\begin{proof}
	¿TRIVIAL?
\end{proof}

\begin{corollary}
	Assuming the usual formulation of the Axiom of Infinity (SEE KUNEN!!!!!), our notation  $\type(A,<) = \omega$ coincides with its usual form stating that $\omega$ is an ordinal and $\langle A,< \rangle$ is order isomorphic to it.
\end{corollary}



%Ver si se puede sacar Infinito sin problemas

%Kunen da la conjunción de Partes o Reemplazo.

\begin{prop}[$\esti$]
	There is a well ordering $\langle A,< \rangle$ such that 
	$\type(A,<) = \omega$.
\end{prop}

\begin{theorem}[$\esti$]
	\label{hartogs_set}
	Let $\langle A,< \rangle$ be a well order such that $\type(A,<) = \omega$, then there is no sequence $\{X_n\}_{n\in A}$ such that for all $n\in A$ ($\mathcal{P}(X_{n+1}) <_c X_n$).
\end{theorem}

Before moving on to the proof, let us note that Definition \ref{aleph} does not rely on any axiom outside of $\estf$, with the caveat that without Replacement, $\aleph(X)$ cannot be proven to be isomorphic to an ordinal.

\begin{proof}
	Assume there is such a sequence, 
	$\mathcal{S} := \{X_n\}_{n\in A}$,
	then $\estf$ proves that $\ran(\mathcal{S})$ exists (\cite[Def. I.6.6]{Kunen}).
	Moreover, for all $X\in\ran(\mathcal{S})$: 
	\[
	\aleph(X) \subset \mathcal{P}(\WO(X)) \subseteq \mathcal{P}(\WO(\union \ran(\mathcal{S}))),
	\]
	where both inclusions follow from Definition \ref{aleph}.
	
	\noindent Thus, for all $X\in\ran(\mathcal{S})$, we have $\aleph(X)\in\mathcal{P}(\mathcal{P}(\WO(\union \ran(\mathcal{S}))))$. Hence, $\esti$ proves the existence of the sequence $\{\aleph(X_n)\}_{n\in A}$, which would of course be trivial using Replacement.
	
	It is straightforward to prove that for each $n\in A$, 
	$\mathcal{P}^{3}(X_{n+3}) <_c X_n$, 
	indeed, applying the hypothesis on $\mathcal{S}$ yields:
	\begin{align*}
		&\mathcal{P}(X_{n+3}) \hspace{4.5 pt} <_c X_{n+2}
		\\ \implies &\mathcal{P}^2(X_{n+3}) \leq_c \hspace{4.5 pt} \mathcal{P}(X_{n+2}) <_c X_{n+1}
		\\ \implies &\mathcal{P}^3(X_{n+3}) \leq_c \mathcal{P}^{2}(X_{n+2}) \leq_c \mathcal{P}(X_{n+1}) <_c X_n.
	\end{align*}
	As a consequence,
	$\aleph(X_{n+3}) <_c \mathcal{P}^{3}(X_{n+3}) <_c X_n$, where the first inequality comes from applying Theorem \ref{partes3}.
	
	\noindent Let us now show that $\aleph(X_{n+3}) <_c \aleph(X_n)$ for all $n\in\omega$. Well ordered sets are pairwise comparable with the relation $\leq_c$, in particular: $\aleph(X_{n+3}) <_c \aleph(X_n)$ or $\aleph(X_n) \leq_c \aleph(X_{n+3})$. The latter statement cannot hold, since it would imply $\aleph(X_n) <_c X_n$, which contradicts Lemma \ref{aleph2}.
	
	Finally, $\{\aleph(X_n)\}_{n\in A}$
	%$\{\aleph(X_{3n})\}_{n\in\omega}$ 
	is a set of well ordered sets that is not well founded with the $<_c$ relation, which cannot exist under $\estf$. Indeed, 
	it is non empty and it has no least element: for any $\aleph(X_n)$, we have $\aleph(X_{n+3}) <_c \aleph(X_n)$.
	
\end{proof}

\begin{notation}
	Given that without the Axiom of Infinity $\omega$ could be a proper class, when we say $n\in\omega$ we shall really mean $\forall x\in n \ (x = \emptyset \lor \exists y \in n \ x = S(y))$, where $S(a) = a \cup \{a\}$.
\end{notation}

The following theorem is equivalent to the previous one when using Replacement but strictly stronger when working within $\estf$, it essentially shows that any formula defining a sequence like the one from (¿LE PONEMOS NOMBRE A ESTA PROPIEDAD?) Theorem \ref{hartogs_set} as a proper class of ordered pairs must be inconsistent with $\estf$.

\begin{theorem}[$\estf$]
	\label{hartogs_class}
	Let $\Phi(\cdot,\cdot)$ be a formula with two free variables, then $\Phi$ cannot satisfy both of the following properties:
	\begin{enumerate}[(i)]
		\item \label{nat_define} For each $n\in\omega$ there is exactly one set $X$ that satisfies $\Phi(n,X)$.
		\item \label{dec_seq} For each $n\in\omega$ the following holds: 
		\[
			(\Phi(n,X) \land \Phi(n+1,Y)) \implies \mathcal{P}(Y) <_c X.
		\] 
	\end{enumerate}
\end{theorem}

Before giving a proof for Theorem \ref{hartogs_class} we note the following result from \cite[Corollary 7.33]{Moscho}, which can be proven in $\estf$.

\begin{prop}\label{class_wo}
	Every non-empty class $\mathcal{C}$ of well ordered sets has a $\leq_o$-least member, i.e., for some $U_0 \in \mathcal{C}$ and all $U\in\mathcal{C}$, $U_0$ is isomorphic to an initial segment of $U$.
\end{prop}

\begin{proof}[Proof of Theorem \ref{hartogs_class}]
	Let $\Phi$ be a formula with two free variables and assume it satisfies (\ref{nat_define}) and (\ref{dec_seq}), we shall take the following class:
	\[
		 \mathcal{C} := \{X: \exists Y ((\exists n\in\omega \ \Phi(n,Y)) \land X = \aleph(Y))\}.
	\]
	By (\ref{nat_define}), Definition \ref{aleph} and Lemma \ref{aleph2}, $\mathcal{C}$ is a non-empty class of well ordered sets. By Proposition \ref{class_wo}, there must be a $\leq_o$-least member of $\mathcal{C}$.
	Let $X_0$ be such a member of $\mathcal{C}$, then there must be some $Y_0$ and $n_0 \in\omega$ such that $X_0 = \aleph(Y_0) \land \Phi(n_0,Y_0)$.
	The assumption (\ref{nat_define}) implies there are $Y_1 , Y_2, Y_3$ such that $\Phi(n_0 + i , Y_i)$ for $i\in\{1,2,3\}$.
	Applying (\ref{dec_seq}), we get:
	
	\begin{align*}
		&\mathcal{P}(Y_3) \hspace{4.5 pt} <_c Y_2
		\\ \implies &\mathcal{P}^2(Y_3) \leq_c \hspace{4.5 pt} \mathcal{P}(Y_2) <_c Y_1
		\\ \implies &\mathcal{P}^3(Y_3) \leq_c \mathcal{P}^{2}(Y_2) \leq_c \mathcal{P}(Y_1) <_c Y_0.
	\end{align*}
	Recall from Theorem \ref{partes3} that $\aleph(Y_3) <_c \mathcal{P}^{3}(Y_3)$, and using the inequality above, we obtain $\aleph(Y_3) <_c Y_0$. But $\aleph(Y_3)\in\mathcal{C}$, which contradicts $X_0$ being $\leq_o$ minimal. 
\end{proof}

DUDA IMPORTANTE, ¿CUÁL ES EL ALEPH DE UN CONJUNTO DEDEKIND FINITO (NO FINITO)? ¿SIEMPRE VA A SER UN BUEN ORDEN FINITO? EN CASO CONTRARIO, EN UN MODELO DE BST SIN EL AXIOMA DE INFINITO TIENE QUE EXISTIR ESE ALEPH, ¿PERO EL HECHO DE QUE HAYA UN BUEN ORDEN INFINITO NO IMPLICA QUE $\omega$ EXISTE?

OTRA DUDA, ¿EN $\estf$ VALE EL SIGUIENTE TEOREMA? ES SIQUIERA ENUNCIABLE?

We would like to prove the following stronger version of Theorem \ref{hartogs_class}.

\begin{theorem}[$\estf$]
	\label{hartogs_inf_classes}
	For every $n\in\mathbb{N}$, let $\Phi_n(\cdot)$ be a formula with one free variable such that for each $n\in\mathbb{N}$, $\estf \vdash \exists! X \ \Phi_n (X)$, then there must be a $k \in \mathbb{N}$ such that 
	\[
		\estf \vdash \exists X,Y ,\, \Phi_k (X) \land \Phi_{k+1} (Y) \land \neg(\mathcal{P}(Y) <_c X)
	\]
\end{theorem}

We would like to show the equivalence of the previous theorem to the following result.

\begin{lemma}
	\label{easier_hartogs_class}
	For every $n\in\mathbb{N}$, let $\Phi_n(\cdot)$ be a formula with one free variable such that for each $n\in\mathbb{N}$, $\estf \vdash \exists! \langle X, < \rangle \ \Phi_n ( \langle X,<\rangle)$, then there must be a $k \in \mathbb{N}$ such that 
	\[
		\estf \vdash \exists X,Y, \, \Phi_k (X) \land \Phi_{k+1} (Y) \land X \leq_o Y
	\]
\end{lemma}

We state a weaker version of Theorem \ref{hartogs_inf_classes}.

\begin{lemma}
	\label{easier_inf_classes}
	For every $n\in\mathbb{N}$, let $\Phi_n(\cdot)$ be a formula with one free variable such that for each $n\in\mathbb{N}$, $\estf \vdash \exists! X \ \Phi_n (X)$, then there must be a $k \in \mathbb{N}$ such that 
	\[
		\estf \vdash \exists X,Y ,\, \Phi_k (X) \land \Phi_{k+1} (Y) \land \neg(\mathcal{P}^{3}(Y) <_c X)
	\]
\end{lemma}

\begin{theorem}
	Lemma \ref{easier_hartogs_class} implies Lemma \ref{easier_inf_classes}.
\end{theorem}

\begin{proof}
	By contradiction, let us assume Lemma \ref{easier_inf_classes} does not hold. Then for every $k\in\mathbb{N}$ there exist two sets $X$ and $Y$ such that
	\[
		\Phi_k (X) \land \Phi_{k+1} (Y) \land \mathcal{P}^{3}(Y) <_c X.
	\]
	Let $\Psi_n (A)$ be the formula $\exists X (\Phi_n (X) \land A = \aleph(X))$.
	Given that each $n\in\mathbb{N}$ defines a unique well ordered set, we can apply Lemma \ref{easier_hartogs_class} with the formulae $\Psi_n$. Let $k$ be a natural number and $A,B$ such that 
	\begin{equation}\tag{1}
		\label{eq_wo}
		\Psi_k (A) \land \Psi_{k+1} (B) \land A \leq_o B
	\end{equation}
	Let $X$ and $Y$ be the unique sets satisfying 
	\begin{align*}
		\Phi_k (X) \land A &= \aleph(X) 
		\\ \Phi_{k+1} (Y) \land B &= \aleph(Y).
	\end{align*}
	By assumption, $\mathcal{P}^{3}(Y) <_c X$; and applying Theorem \ref{partes3} yields
	\begin{equation}\tag{2}
		\label{eq_order}
		\aleph(Y) <_c \mathcal{P}^{3} (Y) <_c X. 
	\end{equation}
	Given that all well orders are pairwise comparable with the $\leq_o$ relation, it must be the case that $\aleph(Y) <_o \aleph(X)$ or $\aleph(X) \leq_o \aleph(Y)$.
	But $\aleph(X) \leq_o \aleph(Y)$, together with \ref{eq_order}, would imply $\aleph(X) <_c X$, which contradicts Lemma \ref{aleph2}.
	
	So the inequality $\aleph(Y) <_o \aleph(X)$  must hold. Recall that $\aleph(Y) = B$ and $\aleph(X) = A$, so the previous inequality means $B <_o A$, which contradicts \ref{eq_wo}.
	
	We have thus derived a contradiction from assuming Lemma \ref{easier_hartogs_class} and the negation of Lemma \ref{easier_inf_classes}.
\end{proof}

\begin{theorem}
	Lemma \ref{easier_inf_classes} implies Theorem \ref{hartogs_inf_classes}.
\end{theorem}
\begin{proof}
	HACER
\end{proof}

We would like to have a unified way of referring to well orderings like $\aleph_1$ or $\aleph_\omega$ in the absence of Replacement. We cannot even prove $\omega+\omega$ exists in ZC (¿AGREGAMOS REFERENCIA A ESTO O SÓLO MENCIONAMOS EL MODELO?), much less the aforementioned cardinals, so we will introduce a notion analogous to Definition \ref{omega_type}, which was a way of talking about well orderings that were "essentially $\omega$".

\begin{definition}
	Let $\langle A,< \rangle$ be a well ordering. Then:
	\begin{enumerate}[(i)]
		\item Let $x\in A$, then $x{\downarrow} := \{y\in A: y<x\}$ and $<_x = \{(a,b) \in x{\downarrow}\times x{\downarrow} : a<b\}$.
		\item $\langle A,< \rangle$ is a \emph{cardinal order} iff for all $x\in A$, $x{\downarrow} <_c A$.
		\item $\langle A,< \rangle$ is an \emph{infinite cardinal order} iff it is a cardinal order and ${\type(A,<)} = \omega$ or there is some $x\in A$ such that $\type(x{\downarrow},<_x) = \omega$.
		\item If $\langle A,< \rangle$ is an infinite cardinal order, we define its cardinal length as $\mathcal{C}(A,<) = \mathcal{C}(A) = \{x\in A: (x{\downarrow},<_x) \text{ is an infinite cardinal order}\}$. We give $\mathcal{C}(A)$ the inherited order from $A$.
	\end{enumerate}
\end{definition}

VER SI EFECTIVAMENTE ES CIERTO LO SIGUIENTE!!!

\begin{lemma}
	\label{weak_aleph}
	Let $A$ and $B$ be two infinite cardinal orders, then $A <_c B$ iff $\mathcal{C}(A) <_o \mathcal{C}(B)$ and $A =_c B$ iff $\mathcal{C}(A) =_o \mathcal{C}(B)$.
\end{lemma}

\begin{lemma}
	For any set $X$, $\aleph(X)$ is a cardinal order. Moreover, if $X$ is not finite, $\aleph(X)$ is an infinite cardinal order.
\end{lemma}

In this manner, any infinite cardinal whose $\mathcal{C}$ has exactly one element behaves like $\aleph_1$, even when the latter does not exist as an ordinal number.

\begin{proof}
	\emph{Sketch of a proof of the equivalence between \ref{hartogs_inf_classes} and \ref{easier_hartogs_class}.}
	
	Given $\Phi_n (\cdot)$, we can take $\Psi_n (X) \equiv \exists Y \ (\Phi_n (Y) \land X = \aleph(Y))$. Using our usual trick of taking the sequence of the alephs, we get a sequence of well ordered sets. Now we apply Lemma \ref{weak_aleph}, which gives us one of the directions of the equivalence, the other should be trivial.
\end{proof}

%Voy agregando bibliografía, con el Kunen y el Jech por si llegan a hacer falta
\bibliographystyle{plain}
\bibliography{bibliografia-hartogs}
\printindex
